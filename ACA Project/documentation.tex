\documentclass[a4paper]{article}

%% Useful packages
\usepackage{graphicx}	
\usepackage[colorlinks=true, allcolors=black]{hyperref}
\usepackage[fontsize=11pt]{scrextend}
\usepackage{titlesec}
\usepackage[section]{placeins}

\setlength\parindent{0pt}
\titleformat*{\section}{\Large\bfseries}
\titleformat*{\subsection}{\Large\bfseries}

\title{ACA Project : Implementation of a subset of OpenCL wrapper for Mango API}

\begin{document}

\begin{titlepage}
\begin{figure}
\centering
\includegraphics[width=0.2\textwidth]{polimi.jpg}
\par
\LARGE Politecnico di Milano
\end{figure}


\maketitle

\raggedright
Authors:
\begin{itemize}
	\item Domenico FAVARO (Mat. 837995)
        	\item Matheus FIM (Mat. 876069)
\end{itemize}
\raggedleft
Prof. Giovanni Agosta
\thispagestyle{empty}
\end{titlepage}

\tableofcontents
\newpage

\section{Mango API - OpenCL}
The Mango API is a tool to manage resources and parallelism that aims to attain higher resource efficiency to be used by HPC (High Performance Computing) applications. 

As OpenCL is a framework to manage resources as devices, CPUs and GPUs, to control the platform and execute programs on these devices. Providing an interface for parallel computing using task- and data-based parallelism.

\section{Objective}

On the implementation of the wrapper for OpenCL compliant code we need to use the Mango API for parallel programming. Calls to functions should be in OpenCL format and then translated wrapped around to its Mango equivalent.
 
The OpenCL functions will be presented with their equivalent Mango function or translated in a way to do what OpenCL wants using the available Mango functions. If none can be achieved then the OpenCL function will be marked as not wrappable.

\section{Mango Elements}
\subsection{Context}
Created automatically when \ttfamily mango\_init()\rmfamily is called
\subsection{KernelFunction}
This is an array of function pointers to support multiple versions of the kernel. Data structure that'll store what program will be executed in which kernel.
\subsection{Kernels}
A kernel represents a function that'll be executed, the concept is similar to OpenCL kernel. A Mango kernel however will be included in a task graph.
\subsection{Buffers}
Used to transfer datao to and from GN (Guest) and HN (Host). Mango implements a further FIFO Buffer for a a burst-mode data transfer.
\subsection{TaskGraph}
Data structure representing an execution (task) that can include one or several kernels defining which buffers and events it will use.
\subsection{Events}
Used to synchronize executions between kernels. A typical event simbolizes the completion of a Kernel.
\subsection{Arguments}
Typically pointers to buffers, with the appropriate size or alternately, they can be scalar values or events to be passed to the kernel.
\subsection{Mango\_Types and Error\_Types}
Define the types that will be used inside Mango (ex. File Types) and more interestingly the Errors that can happen during execution, that are significantly less than OpenCL. 

\section{Mango Program Flow}
An usual program executed in Mango API follows a setup that prepares the elements needed to be executed in parallel by creating one or more kernels to which resources (devices) are allocated.
The flow is the following:
\begin{itemize}
	\item[-] Init Mango: \ttfamily mango\_init() \rmfamily
        	\item[-] Kernel Declarations: \ttfamily mango\_kernelfunction\_init(); mango\_load\_kernel(); mango\_kernel\_t = mango\_register\_kernel();  \rmfamily
	\item[-] Registering buffers: \ttfamily mango\_buffer\_t = mango\_register\_memory() \rmfamily
	\item[-] Create a Task Graph (returns an event):  \ttfamily mango\_task\_graph\_t = mango\_task\_graph\_create() \rmfamily
	\item[-] Resource Allocation: \ttfamily mango\_resource\_allocation() \rmfamily
	\item[-] Declare Arguments that will be passed to the kernel: \ttfamily mango\_arg\_t = mango\_arg() \rmfamily
	\item[-] Transfer buffers from host to device: \ttfamily mango\_write() \rmfamily
	\item[-] Start the kernels (returns an event) and execute synchronization tasks between events: \ttfamily mango\_event\_t = mango\_start\_kernel() \rmfamily
	\item[-] Read the result from device to host: \ttfamily mango\_read() \rmfamily
	\item[-] Deallocate resources, destroy task graph and release Mango: \ttfamily mango\_resource\_deallocation(); mango\_task\_graph\_destroy\_all(); mango\_release(); \rmfamily
	\item[-] Continue offline code... 
\end{itemize} 

\section{OpenCL Elements - Wrapper}
\subsection{Platform}
Not applicable as the platform used will always be Mango.
\subsection{Devices}
Mango = Emulate?
\subsection{Context}
Mango Context will be used.
\subsection{Buffer Objects - Memory Objects}
Equivalent to Buffers, used to read write data.
\subsection{Programs}
Mango = KernelFunction
\subsection{Kernels}
Mango = Kernel
\subsection{Events}
Mango = Events
\subsection{Command Queue}
Mango = TaskGraph?
\subsection{Exceptions}
Mango = ErrorTypes
\subsection{Images}
As the wrapper will limit itself outside of graphic objects, no Images (2D or 3D) will be used.

\section{OpenCL Flow - Wrapper}
OpenCL flow offers three types of task parallelism: 
\begin{itemize}
	\item Internal to the task: won't be addressed directly in the wrapper.
	\item Kernels executing tasks concurrently in an out-of-order queue: This can be wrapped on Mango.
	\item Use of events synchronization: This is done by task graphs in Mango and it provides a set of tools to sync the queues. OpenCL does not address this specifically as it has no concept of Task Graph but the event synchronization concept is the same and thus can be wrapped.
\end{itemize} 
 An example of OpenCL program flow is the following:
\begin{enumerate}
	\item Get available Platform
        	\item Get available Devices
	\item Create Context
	\item Create Command Queue
	\item Create Buffers
	\item Create and Build Program
	\item Create Kernel
	\item Set Kernel Arguments
	\item Queue Buffers
	\item Queue and execute Kernels
	\item Read the result from read buffer
	\item Release all resources, program, kernel, buffers and context
	\item Continue offline code... 
\end{enumerate}
We'll present the wrapping for each of these steps.

\subsection{Get available Platform}
No
\subsection{Get available Devices}
???
\subsection{Create Context}
\ttfamily mango\_init() \rmfamily
\subsection{Create Command Queue}
\ttfamily mango\_task\_graph\_t = mango\_task\_graph\_create() \rmfamily
\subsection{Create Buffers}
\ttfamily mango\_buffer\_t = mango\_register\_memory() \rmfamily
\subsection{Create and Build Program}
\ttfamily mango\_kernelfunction\_init(); 
mango\_load\_kernel(); \rmfamily
\subsection{Create Kernel}
\ttfamily mango\_kernel\_t = mango\_register\_kernel();  \rmfamily
\subsection{Set Kernel Arguments}
\ttfamily mango\_arg\_t = mango\_arg() \rmfamily
\subsection{Queue Buffers}
\ttfamily mango\_write() \rmfamily
\subsection{Queue and execute Kernels}
\ttfamily mango\_event\_t = mango\_start\_kernel() \rmfamily
\subsection{Read the result from read buffer}
\ttfamily mango\_read() \rmfamily
\subsection{Release all resources, program, kernel, buffers and context}
\ttfamily mango\_resource\_deallocation(); 
mango\_task\_graph\_destroy\_all(); 
mango\_release(); \rmfamily

\newpage
\section{Documentation}
\begin{itemize}
	\item \textbf{Mango:}\url{ http://www.mango-project.eu/}
	\item \textbf{OpenCL:} OpenCL Programming Guide - by Benedict Gaster and Timothy G. Mattson
	\item \textbf{Khronos OpenCL:} \url{https://www.khronos.org/opencl/}
\end{itemize}
\newpage


\section{Changelog}
\begin{itemize}
\item \textbf {Version 1.1:} 30/06/2017
\end{itemize}

\end{document}