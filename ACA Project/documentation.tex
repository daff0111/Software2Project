\documentclass[a4paper]{article}

%% Useful packages
\usepackage{graphicx}	
\usepackage[colorlinks=true, allcolors=black]{hyperref}
\usepackage[fontsize=11pt]{scrextend}
\usepackage{titlesec}
\usepackage[section]{placeins}

\setlength\parindent{0pt}
\titleformat*{\section}{\Large\bfseries}
\titleformat*{\subsection}{\Large\bfseries}

\title{ACA Project : Implementation of a subset of OpenCL wrapper for Mango API}

\begin{document}

\begin{titlepage}
\begin{figure}
\centering
\includegraphics[width=0.2\textwidth]{polimi.jpg}
\par
\LARGE Politecnico di Milano
\end{figure}


\maketitle

\raggedright
Authors:
\begin{itemize}
	\item Domenico FAVARO (Mat. 837995)
        	\item Matheus FIM (Mat. 876069)
\end{itemize}
\raggedleft
Prof. Giovanni Agosta
\thispagestyle{empty}
\end{titlepage}

\tableofcontents
\newpage

\section{Mango API - OpenCL}
The Mango API is a tool to manage resources and parallelism that aims to attain higher resource efficiency to be used by HPC (High Performance Computing) applications. 

As OpenCL is a framework to manage resources as devices, CPUs and GPUs, to control the platform and execute programs on these devices. Providing an interface for parallel computing using task- and data-based parallelism.

\section{Objective}

On the implementation of the wrapper for OpenCL compliant code we need to use the Mango API for parallel programming. Calls to functions should be in OpenCL format and then translated wrapped around to its Mango equivalent.
 
The OpenCL functions will be presented with their equivalent Mango function or translated in a way to do what OpenCL wants using the available Mango functions. If none can be achieved then the OpenCL function will be marked as not wrappable.

\section{Mango Elements}
\subsection{Context}
Created automatically when \ttfamily mango\_init()\rmfamily is called. Class that holds the current state of the host-side runtime.
\subsection{KernelFunction}
This is an array of function pointers to support multiple versions of the kernel. Data structure that'll store what program will be executed in which kernel.
\subsection{Kernels}
A kernel represents a function that'll be executed, the concept is similar to OpenCL kernel. A Mango kernel however will be included in a task graph.
\subsection{Buffers}
Used to transfer datao to and from GN (Guest) and HN (Host). Mango implements a further FIFO Buffer for a a burst-mode data transfer.
\subsection{TaskGraph}
Data structure representing an execution (task) that can include one or several kernels defining which buffers and events it will use.
\subsection{Events}
Used to synchronize executions between kernels. A typical event simbolizes the completion of a Kernel.
\subsection{Arguments}
Typically pointers to buffers, with the appropriate size or alternately, they can be scalar values or events to be passed to the kernel.
\subsection{Mango\_Types and Error\_Types}
Define the types that will be used inside Mango (ex. File Types) and more interestingly the Errors that can happen during execution, that are significantly less than OpenCL. 

\section{Mango Program Flow}
An usual program executed in Mango API follows a setup that prepares the elements needed to be executed in parallel by creating one or more kernels to which resources (devices) are allocated.
The flow is the following:
\begin{itemize}
	\item[-] Init Mango: \ttfamily mango\_init() \rmfamily
        	\item[-] Kernel Declarations: \ttfamily mango\_kernelfunction\_init(); mango\_load\_kernel(); mango\_kernel\_t = mango\_register\_kernel();  \rmfamily
	\item[-] Registering buffers: \ttfamily mango\_buffer\_t = mango\_register\_memory() \rmfamily
	\item[-] Create a Task Graph (returns an event):  \ttfamily mango\_task\_graph\_t = mango\_task\_graph\_create() \rmfamily
	\item[-] Resource Allocation: \ttfamily mango\_resource\_allocation() \rmfamily
	\item[-] Declare Arguments that will be passed to the kernel: \ttfamily mango\_arg\_t = mango\_arg() \rmfamily
	\item[-] Transfer buffers from host to device: \ttfamily mango\_write() \rmfamily
	\item[-] Start the kernels (returns an event) and execute synchronization tasks between events: \ttfamily mango\_event\_t = mango\_start\_kernel() \rmfamily
	\item[-] Read the result from device to host: \ttfamily mango\_read() \rmfamily
	\item[-] Deallocate resources, destroy task graph and release Mango: \ttfamily mango\_resource\_deallocation(); mango\_task\_graph\_destroy\_all(); mango\_release(); \rmfamily
	\item[-] Continue offline code... 
\end{itemize} 

\section{OpenCL Elements - Wrapper}
\subsection{Platform}
Mango does not handle platform level, this is not applicable as the platform used will always be Mango itself.
\subsection{Devices}
Similar to the Platform, Mango has no handling of different devices, an emulation can be put in place to present the availability of more than one CPU, however usually OpenCL runs with the first available device and this can be translated to always a CPU available. No handling of GPUs will be made.
\subsection{Context}
Mango Context will be used.
\subsection{Buffer Objects - Memory Objects}
Equivalent to Mango Buffers, used to read write data, also the way they're set are similar.
\subsection{Programs}
Program contains the kernel and define the function that it'll execute so is paired to Mango's KernelFunction.
\subsection{Kernels}
It'll work the same as Mango Kernels however as the method of setting arguments, the wrapper must keep track of the kernels arguments to update the Mango kernel argument each time an OpenCL call for setting them is made.
\subsection{Events}
The OpenCL events function in parallel with Mango events to synchronize between kernels, usually at completion.
\subsection{Command Queue}
Slightly different to Mango's TaskGraph the way kernels are organized inside the devices is with Command Queues, inside of which every function is executed in order of the queue. But the concept of a structure that holds the kernels to be executed still holds.
\subsection{Exceptions}
Mango has less Error codes than OpenCL and most of them are translated evenly, which works nicely in case of any error is presented in the Mango Platform, its OpenCL error code can be shown.
\subsection{Images}
As the wrapper will limit itself outside of graphic objects, no Images (2D or 3D) will be used.
For a basic OpenCL program all these elements are the only thing needed so the wrapper will be limited itself to this.

\section{OpenCL Flow - Wrapper}
OpenCL flow offers three types of task parallelism: 
\begin{itemize}
	\item Internal to the task: won't be addressed directly in the wrapper.
	\item Kernels executing tasks concurrently in an out-of-order queue: This can be wrapped on Mango.
	\item Use of events synchronization: This is done by task graphs in Mango and it provides a set of tools to sync the queues. OpenCL does not address this specifically as it has no concept of Task Graph but the event synchronization concept is the same and thus can be wrapped.
\end{itemize} 
 An example of OpenCL program flow is the following:
\begin{enumerate}
	\item Get available Platform
        	\item Get available Devices
	\item Create Context
	\item Create Command Queue
	\item Create Buffers
	\item Create and Build Program
	\item Create Kernel
	\item Set Kernel Arguments
	\item Queue Buffers
	\item Queue and execute Kernels
	\item Read the result from read buffer
	\item Release all resources, program, kernel, buffers and context
\end{enumerate}
We'll present the wrapping for each of these steps.

\subsection{Get available Platform}
OpenCL \ttfamily clGetPlatformIDs() \rmfamily will always return one available platform.
\subsection{Get available Devices}
OpenCL \ttfamily  clGetDeviceIDs() \rmfamily will return one available device.
\subsection{Create Context}
OpenCL \ttfamily clCreateContext() \rmfamily will translate to \ttfamily mango\_init() \rmfamily and the context will be the one created by Mango and will be treated as a CPU context.
\subsection{Create Command Queue}
The command queue creation translates differently to Mango as in mango the Task graph is created after all kernels are created, while OpenCL creates first the Queue and then 'queues' the kernels in it when they're going to be executed, however the data structure can be created (empty) when  \ttfamily clCreateCommandQueue() \rmfamily is called using 
\ttfamily mango\_task\_graph\_t = mango\_task\_graph\_create() \rmfamily and then be reused with the selected kernels when it has to queue them.
\subsection{Create Buffers}
OpenCL \ttfamily clCreateBuffer() \rmfamily will be \ttfamily mango\_buffer\_t = mango\_register\_memory() \rmfamily
\subsection{Create and Build Program}
OpenCL \ttfamily clCreateProgramWithSource() \rmfamily will call 
\ttfamily mango\_kernelfunction\_init() \rmfamily and \ttfamily mango\_load\_kernel() \rmfamily
As both load the program that will be running in the kernel.
\subsection{Create Kernel}
OpenCL simple \ttfamily clCreateKernel()\rmfamily will be \ttfamily mango\_kernel\_t = mango\_register\_kernel();  \rmfamily
\subsection{Set Kernel Arguments}
OpenCL \ttfamily clSetKernelArg() \rmfamily functions the same as \ttfamily mango\_arg\_t = mango\_arg() \rmfamily however, when OpenCL sets the arguments it adds it to the kernel in every call, while mango calls a \ttfamily mango\_set\_args() \rmfamily passing the desired kernel all the arguments it'll use. So for each call in OpenCL a set kernel arguments for Mango must be called to keep the list updated.
\subsection{Queue Buffers}
OpenCL \ttfamily clEnqueueWriteBuffer() \rmfamily will be wrapped over \ttfamily mango\_write() \rmfamily basically doing the same function writing a variable in a buffer.
\subsection{Queue and execute Kernels}
OpenCL \ttfamily clEnqueueNDRangeKernel() \rmfamily will queue and execute the kernel so it compares to \ttfamily mango\_event\_t = mango\_start\_kernel() \rmfamily
\subsection{Read the result from read buffer}
OpenCL \ttfamily clEnqueueReadBuffer() \rmfamily will be \ttfamily mango\_read() \rmfamily reading the result after the execution.
\subsection{Release all resources, program, kernel, buffers and context}
OpenCL releases every class separately with \ttfamily clReleaseProgram(); clReleaseKernel(); clReleaseMemObject();\rmfamily etc. while mango uses fewer commands
\ttfamily mango\_resource\_deallocation(); 
mango\_task\_graph\_destroy\_all(); 
mango\_release(); \rmfamily this will be taken in account however individual kernel and buffer deallocation are also implemented in mango as \ttfamily mango\_deregister\_kernel(); mango\_deregister\_memory(); \rmfamily and \ttfamily mango\_deregister\_event.
\rmfamily

\newpage
\section{Documentation}
\begin{itemize}
	\item \textbf{Mango:}\url{ http://www.mango-project.eu/}
	\item \textbf{OpenCL:} OpenCL Programming Guide - by Benedict Gaster and Timothy G. Mattson
	\item \textbf{Khronos OpenCL:} \url{https://www.khronos.org/opencl/}
\end{itemize}
\newpage


\section{Changelog}
\begin{itemize}
\item \textbf {Version 1.1:} 30/06/2017
\end{itemize}

\end{document}